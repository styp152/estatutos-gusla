    \subsection{Decima Octava}

      Es la Instancia Ejecutiva de la Asociaci\'on, tendr\'a la cualidad de ser
      la administradora y representante legal de la \emph{Asociaci\'on Civil
      gUsLA}. La Coordinaci\'on General estar\'a compuesta por siete
      (7) miembros, quienes tendr\'an una duraci\'on de un (1) a\~no en el
      ejercicio de sus funciones y podr\'an ser reelegidos por un m\'aximo de
      dos mandatos adicionales, no obstante deber\'an permanecer en el ejercicio
      de sus cargos y continuar en los mismos hasta que sean legalmente
      reemplazados. La falta absoluta o temporal de un directivo ser\'a suplida
      por un miembro activo designado por la mayor\'i{}a de los otros
      directores, reunidos para tal efecto. Los integrantes de la Coordinaci\'on
      General desempe\~naran este cargo en forma ad-honoren.

    \subsection{Decima Novena: Atribuciones de la Coordinaci\'on General}

      Se reunir\'a obligatoriamente una (1) vez cada tres (3) meses y con la
      frecuencia que sea necesaria, para el mejor desempe\~no de sus funciones y
      de los objetivos de la Asociaci\'on, establecidos en sus Estatutos
      Sociales. La convocatoria ser\'a comunicada por v\'ia escrita y con un
      plazo de antelaci\'on de siete (7) d\'ias. Sus decisiones se tomar\'an por
      votaci\'on simple de la mitad m\'as uno, es decir, de la aprobaci\'on de
      cuatro (4) directivos de la Coordinaci\'on General; entre otras
      \emph{Atribuciones y Responsabilidades}.

      \begin{enumerate}

        \item
            Implantar las l\'i{}neas de trabajo dadas por las Asambleas o por
            car\'acter de urgencia tomadas por la Coordinaci\'on General. 
        \item    
            Velar por la buena marcha y organizaci\'on diaria de la
            Asociaci\'on, as\'i{} como examinar cuando lo estime conveniente,
            las cuentas y documentos de la Asociaci\'on.
        \item
            Dictar los reglamentos para el uso de los bienes o instalaciones de
            la instituci\'on o para la determinaci\'on de las contribuciones que
            deban hacer los miembros para el cumplimiento de los objetivos de la
            Asociaci\'on, as\'i{} como reformar total o parcialmente dichos
            reglamentos. 
        \item
            Celebrar, firmar contratos, convenios necesarios o convenientes para
            el cumplimiento de los objetivos de la Asociaci\'on y presentar el
            informe Anual respectivo a la Asamblea General.
        \item
            Examinar los Balances y Estados de Cuenta e informar a la Asamblea
            General del resultado de su an\'alisis incluyendo sus observaciones
            y proposiciones a que haya lugar. 
        \item
            Recomendar a la Asamblea General la contrataci\'on de personal
            t\'ecnico, profesional para el buen funcionamiento de la
            Asociaci\'on.
        \item
            Autorizar los gastos no contemplados en el presupuesto a trav\'es de
            un informe detallado para poder disponer y la manera como deben ser
            cubiertos.
        \item
            Podr\'a adquirir bienes muebles o inmuebles, Constituir
            garant\'i{}as sobre bienes de la Asociaci\'on. Solicitar dinero en
            calidad de pr\'estamo.
        \item
            Elaborar el Reglamento de R\'egimen Interior que ser\'a aprobado por
            la Asamblea General.
        \item
            Certificar el cumplimiento de los requisitos exigidos a aquellos que
            soliciten el ingreso en la Asociaci\'on.
        \item
            Mantener el equipamiento para alguna actividad de la Asociaci\'on.
        \item
            Cualquier otra facultad que no sea de la exclusiva competencia de la
            Asamblea General de miembros.

      \end{enumerate}

    \subsection{Vigesima: La Coordinaci\'on General. Duraci\'on, Atribuciones y
    Organizaci\'on}

      La duraci\'on de los integrantes de la Coordinaci\'on General en sus
      funciones ser\'a de un a\~no, pudiendo ser reelegidos por un m\'aximo de
      dos mandatos consecutivos. Atribuciones: 
      
        \begin{enumerate}
          \item
            La gesti\'on de la coordinaci\'on diaria de las actividades y
            operaciones de la asociaci\'on.
            
          \item
            Ser representante legal de la Asociaci\'on ante cualquier persona
            natural o jur\'i{}dica de car\'acter publico o privado, nacional,
            internacional o judicial.             

          \item
            Firmar contratos y/o convenios en conjunto con el Coordinador de
            Finanzas ante instituciones publicas o privadas. 
          
          \item
            Convocar las reuniones de la Coordinación General y Asamblea.

          \item
            Abrir y movilizar cuentas bancarias a nombre de la Asociaci\'on en
            conjunto con el Coordinador de Finanzas. 
            
          \item
            Firmar en nombre de la Asociaci\'on los documentos a ser registrados
            o autenticados. Organizaci\'on: Estar\'a conformada la
            Coordinaci\'on General por siete (7) miembros que se denominar\'an
            coordinadores y que tendr\'an sus funciones espec\'i{}ficas al cargo
            que ocupan.
        
        \end{enumerate}

    \subsection{Vigesima Primera}
      El Coordinador de Secretaria tendr\'a a su cargo la direcci\'on de los
      trabajos puramente administrativos de la Asociaci\'on, expedir\'a
      certificaciones, llevar\'a los archivos y custodiar\'a la documentaci\'on
      de la entidad, haciendo que se cursen a la Autoridad las comunicaciones
      sobre designaci\'on de Coordinaciones Generales, celebraci\'on de
      Asambleas y aprobaci\'on de los presupuestos y estado de cuentas. Asimismo
      tambi\'en deber\'a:

        \begin{enumerate}
          
          \item
            Custodiar los libros y documentos, excepto los de Contabilidad,
            as\'i{} como el archivo.

          \item
            Levantar las actas de las reuniones, tanto de la Coordinaci\'on
            General como de la Asamblea General, firmado por los presentes, de
            todo los tratado y decidido en la reuniones.

          \item
            Asistir a la Coordinaci\'on General para redactar el \emph{Orden del
            D\'i{}a} y cursar las convocatorias.

          \item
            Redactar la Memoria Anual de la Asociaci\'on.

          \item
            Llevar al d\'i{}a el archivo con los nombres y datos de los socios
            afilidados, as\'i{} como las altas y las bajas de los mismos.

          \item
            Ejecutar los acuerdos estatutarios adoptados, bajo la supervisi\'on
            del Coordinador General.

          \item
            Custodiar las firmas electr\'onicas que posea la Asociaci\'on.

          \item
            Llevar y suscribir un archivo de la correspondencia enviada y
            recibida.

        \end{enumerate}

    \subsection{Vigesima Segunda}
      El Coordinador de Finanzas tendr\'a bajo su responsabilidad el
      funcionamiento econ\'omico de la Asociaci\'on, y le corresponde:
      
        \begin{enumerate}
          \item
            Custodiar los fondos de la Asociaci\'on, respondiendo de las
            cantidades de que se haya hecho cargo, conservando en Caja aquellas
            que la Coordinaci\'on General estime oportunas para el
            desenvolvimiento normal de la Asociaci\'on, ingresando los dem\'as
            en entidades financieras, cuentas de dep\'osito, ahorro, o en cuenta
            corriente de la que no se podr\'an extraer fondos salvo mediante
            cheque o transferencia autorizada por dos de las siguientes
            personas: Coordinador General, Coordinador de Finanzas u otra
            persona apoderada por la Coordinaci\'on General. 
            
          \item
            Hacerse cargo de las cantidades que ingrese la Asociaci\'on y
            archivar los libramientos que se hagan efectivos, con sus
            justificantes. 
            
          \item
            Recaudar los fondos de la entidad, custodiarlos e invertirlos como
            lo determine la Coordinaci\'on General. 
            
          \item
            Efectuar los pagos, con el visto bueno del Coordinador General. 
            
          \item
            Dirigir la contabilidad. 
            
          \item
            Llevar el libro de Estados de Cuentas, con las indicaciones de
            ingresos, gastos y saldo. Adem\'as de la hoja de cuentas corrientes
            con la entidad financiera respectiva. 
            
          \item
            Confeccionar el estado de cuentas anual. 
            
          \item
            Entregar, dentro de los diez \'ultimos d\'i{}as de cada trimestre,
            al Coordinador General, un extracto de pagos e ingresos habidos en
            el trimestre. 
          
          \item
            Intervenir todas las operaciones contables de la Asociaci\'on,
            revisando e informando de todas las cuentas rendidas.

          \item
            Elaborar propuestas financieras de proyectos que beneficien a la
            Asociaci\'on.

        \end{enumerate}
        
    \subsection{Vigesima Tercera}
      El Coordinador de Relaciones Institucionales tendr\'a las siguientes
      atribuciones: Asesorar, colaborar, convenir, articular y gestionar
      acuerdos y convenios con aquellos organismos p\'ublicos y privados, grupos
      e instituciones, nacionales, o internacionales y multilaterales, para el
      desarrollo de programas que est\'en enmarcados en los anteriores
      objetivos.

    \subsection{Vigesima Cuarta}
      El Coordinador de Protocolo y Log\'i{}stica tendr\'a entre sus
      atribuciones: Atender, organizar, coordinar, presentar presupuestos para
      el desarrollo de los eventos y actividades propias de la Asociaci\'on.
    
    \subsection{Vigesima Quinta}
       Al Coordinador General le corresponde:

        \begin{enumerate}
          \item
            Representar legalmente a la Asociaci\'on ante toda clase de
            organismos e instituciones, publicas, semiprivadas o privadas,
            locales, municipales, estadales, nacionales, internacionales,
            interinstitucionales 
          
          \item
            Convocar, presidir y levantar las sesiones que celebren la
            Coordinaci\'on General y la Asamblea General, dirigiendo las
            deliberaciones de una y otra. 
          
          \item
            Cumplir y hacer cumplir los acuerdos tomados por la Coordinaci\'on
            General y la Asamblea General. 
            
          \item
            Ordenar pagos y autorizar con su firma los documentos, actas y
            correspondencia de la Asociaci\'on. 
            
          \item
            Adoptar cualquier medida urgente que la buena marcha de la
            Asociaci\'on aconseje o en el desarrollo de sus actividades resulte
            necesaria o conveniente, sin perjuicio de dar cuenta posteriormente
            a la Coordinaci\'on General. 
      
          \item
            Velar y firmar conjuntamente con el Coordinador de Finanzas, todo lo
            relacionado al sistema contable.  
      
          \item
            Aprobar conjuntamente con las distintas Coordinaciones las
            actividades inherentes a cada una de ellas.

        \end{enumerate}

    \subsection{Vigesima Sexta}
      Han sido elegidos para formar parte de la Coordinaci\'on General para el
      periodo \emph{x} al \emph{y}, los siguientes ciudadanos: X titular de la
      c\'edula de identidad \textnumero \ldots, como Coordinador General; Y,
      titular de la c\'edula de identidad \textnumero \ldots, como Coordinador
      de Secretaria; Z, titular de la c\'edula de identidad \textnumero \ldots
      como Coordinador de Finanzas; A titular de la c\'edula de identidad
      \textnumero \ldots, como Coordinador de Relaciones Institucionales; \ldots
