    \subsection{Decima Segunda}

      La Asamblea General de Miembros v\'alidamente constituida es la m\'axima
      instancia de control y toma de decisiones y las mismas ser\'an de
      obligatorio cumplimiento para todos sus miembros, incluso para quienes no
      hubieren concurrido a ellas.

    \subsection{Decima Tercera}
        
      La Asamblea General de miembros se reunir\'a con car\'acter Ordinario al
      menos una vez al a\~no; con car\'acter Extraordinario cuantas veces
      amerite o cuando as\i{} lo solicite al menos una tercera parte de los
      miembros activos o la Coordinaci\'on General.

    \subsection{Decima Cuarta}

      El qu\'orum para las Asambleas será v\'alido con la presencia de un tercio
      de sus miembros, de no verificarse el qu\'orum, la Asamblea ser\'a
      convocada para las veinticuatro (24) horas siguientes en cuyo caso el
      qu\'orum se constituir\'a con los presentes. Las decisiones se tomar\'an
      con la mayor\i{}a simple, excepto en los casos expresamente tipificados en
      los Estatutos Sociales.

    \subsection{Decima Quinta}

      Las Asambleas Ordinarias de miembros ser\'an celebradas en el primer
      trimestre del a\~no y convocadas por la Coordinaci\'on General mediante
      comunicaci\'on escrita, expresando lugar, fecha y hora de la reuni\'on,
      con quince d\'i{}as  de anticipaci\'on. Las Asambleas Extraordinarias
      podr\'an ser convocadas por la Coordinaci\'on General o a petici\'on
      escrita por al menos una tercera parte de los miembros activos de la
      Asociaci\'on. Para las Asambleas Extraordinaria, la convocatoria deber\'a
      incluir la agenda a tratar y la Asamblea se limitar\'a a tratar solo los
      puntos para las cuales fue convocada.

    \subsection{Decima Sexta}
      
      Las Asambleas de miembros ser\'a presidida y coordinada, por la junta
      directiva que es la Coordinaci\'on General, la cual tendr\'a la
      responsabilidad de convocarla y proponer la agenda, facilitar su
      desarrollo y hacer las decisiones, acordadas por la mayor\i{}a de los
      votos. De toda Asamblea, la secretaria o secretario de la Coordinaci\'on
      General levantar\'a un Acta, en la que se har\'a constar los nombres de
      los asistentes, las decisiones y disposiciones acordadas, debiendo ser
      firmada por todos y cada uno de los presentes en la reuni\'on de la
      asamblea.

    \subsection{Decima Septima: Atribuciones de la Asamblea General}

      Son las siguientes, entre otras:

      \begin{enumerate}
        \item 
          Elegir o reelegir los miembros de la Coordinaci\'on General quienes
          durar\'an en sus funciones por un per\'i{}odo por dos (2) a\~nos,
          pudiendo ser reelegido inmediatamente para un (1) periodo igual,
          despu\'es de relaci\'on debe dejar pasar un per\'i{}odo para volver
          ser electo.

        \item
          Aprobar o improbar el presupuesto y Plan Anual, las resoluciones que
          con car\'acter de urgencia hubiere adoptado la Coordinaci\'on General.

        \item 
          Decidir y administrar los bienes de la Asociaci\'on.

        \item 
          Definir las pol\'i{}ticas y estrategias que orienten el trabajo de la
          Asociaci\'on.

        \item 
          Discutir y decidir la ejecuci\'on de proyectos que tiendan al logro de
          los objetivos de la Asociaci\'on.

        \item 
          Decidir sobre las alternativas y los mecanismos para la consecuci\'on
          de recursos que garanticen el logro de los objetivos de la
          Asociaci\'on.

        \item 
          Refrendar sobre la incorporaci\'on y desincorporaci\'on de los
          miembros de la Asociaci\'on. 

        \item 
          Rerformar parcialmente los Estatutos Sociales.

        \item 
          Aprobar la disoluci\'on de la Asociaci\'on.

        \item 
          Acordar los mecanismos o medios de cooperaci\'on e intercambio con los
          organismos e instituciones, gubernamentales o no, que persigan fines
          similares a los de la Asociaci\'on.

        \item 
          Nombrar una comisi\'on que conjuntamente con la Coordinaci\'on
          General, redactar\'an el regalmento interno, aprobando y ampliando lo
          no previsto en los presentes Estatutos Sociales

        \item 
          Aprobar o desaprobar el informe financiero y program\'atico de
          Coordinaci\'on General.

        \item
          Aprobar o desaprobar la constituci\'on de una Federaci\'on, nacional o
          internacional, o su integraci\'on en ella si ya existiese.

      \end{enumerate}
