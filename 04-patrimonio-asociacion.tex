      \subsection{Sexta}
        El patrimonio provendr\'a de:
        
        \begin{enumerate}
          
          \item 
            \label{enum:donacion}De cualesquiera donaci\'on o subsidio que
            reciba de los asociados, personas naturales o jur\'i{}dicas, de
            entes p\'ublicos o privados, nacionales o internacionales.

          \item 
            De los ingresos que se obtengan por gestiones que realice y en
            general del producto de sus actividades, as\'i{} como de convenios
            interinstitucionales a nivel nacional e internacional.
          
          \item 
            Tambi\'en su patrimonio estar\'a constituido por los bienes muebles
            e inmuebles que adquiera a cualquier t\'i{}tulo. 

        \end{enumerate}

        El patrimonio que se trata en el literal \ref{enum:donacion} de esta
        cl\'ausula, ingresar\'a al patrimonio com\'un de la asociaci\'on, a
        menos que por voluntad del aportante o donante tengan destinaci\'on
        espec\'i{}fica.

      \subsection{Septima}
        Los bienes muebles o inmuebles, que ingresen a la \emph{Asociaci\'on
        Civil gUsLA}, se destinar\'an \'unica y exclusivamente al
        cumplimiento de sus objetivos y no podr\'an destinarse a usos distintos
        bajo ninguna circunstancia o concepto, salvo las excepciones que
        pudiesen estar reservadas en la Ley.

      \subsection{Octava}
        La asociaci\'on por medio de su \'organo ejecutivo que es la
        Coordinaci\'on General, podr\'a: Nombrar y revocar asesores y
        apoderados, asesorar, colaborar, convenir y firmar acuerdos y convenios
        con aquellos organismos p\'ublicos y privados, grupos e instituciones
        nacionales e internacionales multilaterales, para el desarrollo de
        programas que est\'en enmarcados en los anteriores objetivos.

      \subsection{Novena}
        Podr\'an pertenecer a la \emph{Asociaci\'on Civil gUsLA}
        
        \begin{enumerate}

          \item 
            Personas, menores y mayores de edad, de nacionalidad venezolana o
            residentes en Venezuela que hayan manifestado su voluntad de
            participar en la Asociaci\'on y que est\'en de acuerdo con el objeto
            y la corporaci\'on de esta Asociaci\'on.

          \item 
            Aquellas personas jur\'i{}dicas, p\'ublicas o privadas, que hayan
            manifestado su voluntad de participar en la Asociaci\'on y que
            est\'en de acuerdo con el objeto y la corporaci\'on de esta
            Asociaci\'on.

        \end{enumerate}

      \subsection{Decima}
        Son atribuciones de todos los miembros.

        \begin{enumerate}

          \item 
            Actuar de conformidad con el Acta Constitutiva y Estatutaria,
            dem\'as reglamentos y acuerdos que dicte la Asociaci\'on.
          
          \item 
            Asistir regularmente a las reuniones convocadas por la Asociaci\'on.
          
          \item
            Tomar parte en cuantas actividades organice la Asociaci\'on en
            cumplimiento de sus fines.

          \item
            Participar con voz y voto en las Asambleas Generales.

          \item 
            Elegir y ser elegidos para cargos directivos.

          \item
            Recibir informaci\'on sobre el desarrollo de la Asociaci\'on as\'i{}
            como de los acuerdos adoptados por los \'organos directivos de la
            Asociaci\'on.

          \item
            Realizar sugerencias a los miembros de la Coordinaci\'on General
            para el mejor cumplimiento de los fines de la Asociaci\'on.

          \item
            Utilizar los recursos de la Asociaci\'on, bajo las condiciones que
            determine la Coordinaci\'on General.

          \item
            Solicitar a la Coordinaci\'on General cuantas aclaraciones e
            informes estimen convenientes sobre el estado de administraci\'on y
            contabilidad de la Asociaci\'on, mediante instancia razonada y/o
            suscrita por el diez por ciento (10\%) de los miembros o
            directamente por cualquier de ellos en las reuniones de la Asamblea
            General.

          \item
            Abonar las cuotas que se fijen, en caso de ser mayores de edad.

          \item
            Participar en las actividades de la Asociaci\'on y trabajar para el
            logro de sus fines.

          \item
            Desempe\~nar, en su caso, las obligaciones inherentes al cargo que
            ocupen.

          \item
            Abstenerse de practicar actividades perjudiciales para la
            Asociaci\'on, dentro o fuera de ella, o que de alguna forma
            obstaculice el cumplimiento de los fines que son propios.

          \item 
            Contribuir con su comportamiento al buen nombre y prestigio de la
            Asociaci\'on.

          \item 
            El car\'acter de miembro se pierde:

            \begin{enumerate}

              \item 
                Por voluntad expresa del miembro, la cual deber\'a notificarse
                por escrito a la Coordinaci\'on General.
              
              \item 
                Por incumplimiento de responsabilidades propias de su
                membres\'ia. \item \label{enum:traidor} Por intervenir o
                participar en actividades contrarias a los principios y
                objetivos de la Asociaci\'on y estos estatutos. Ser
                part\'i{}cipe en el desprestigio de la Asociaci\'on con hechos o
                palabras que perturben gravemente los actos organizados por la
                misma y la normal convivencia entre los asociados, o por
                cualquier otro incumplimiento de las obligaciones inherentes a
                la condici\'on de socio.

              \item
                \label{enum:ladron} Por el uso indebido de fondos y bienes de la
                Asociaci\'on, con prop\'ositos diferentes a sus objetivos en
                beneficio particular o en contravenci\'on a las disposiciones
                reglamentarias y/o estatutarias.
              
              \item 
                Por la ausencia injustificada durante seis (6) meses
                consecutivos de las actividades que delegue o le encomiende la
                Coordinaci\'on General y despu\'es ratificada por la Asamblea
                General de miembros.
                
              \item 
                Por decisi\'on de la Coordinaci\'on General o la Asamblea
                Extraordinaria.

            \end{enumerate}
      
        \end{enumerate}
        
        \subsubsection{Paragrafo Unico}
          En caso de incurrir en faltas graves referidas en los literales
          \ref{enum:traidor} y \ref{enum:ladron} del causal de perder el
          car\'acter de miembro del presente estatuto, la Coordinaci\'on General
          podr\'a acordar la suspensi\'on inmediata de cualquier miembro y
          recomendar su desincorporaci\'on de la Asociaci\'on en la pr\'oxima
          Asamblea la cual deber\'a tomar una decisi\'on final con el voto
          favorable de la mitad m\'as uno de lo miembros activos presentes. La
          exclusi\'on o retiro de un miembro no pone fin a la Asociaci\'on que
          continuar\'a con los otros.

      \subsection{Decima Primera}

        La cualidad de miembro o su retiro de la asociaci\'on, constar\'a
        expresamente en el registro f\'isico o telem\'atico de la Asociaci\'on
        la cual ser\'a intransmisible. La \emph{Asociaci\'on Civil gUsLA}
        no otorgar\'a a sus titulares derechos de participaci\'on
        sobre el su patrimonio, ni sobre sus utilidades, beneficios o p\'erdidas
        de la misma. 
